\documentclass[twocolumn,10pt]{article}
\usepackage[utf8]{inputenc}
\usepackage[margin=0.75in]{geometry}
\usepackage{graphicx}
\usepackage{booktabs}
\usepackage{amsmath}
\usepackage{amssymb}
\usepackage{hyperref}
\usepackage{xcolor}
\usepackage{listings}
\usepackage{float}
\usepackage{multicol}
\usepackage{caption}

\definecolor{codegreen}{rgb}{0,0.6,0}
\definecolor{codegray}{rgb}{0.5,0.5,0.5}
\definecolor{codepurple}{rgb}{0.58,0,0.82}
\definecolor{backcolour}{rgb}{0.95,0.95,0.92}

\lstdefinestyle{mystyle}{
    backgroundcolor=\color{backcolour},
    commentstyle=\color{codegreen},
    keywordstyle=\color{blue},
    numberstyle=\tiny\color{codegray},
    stringstyle=\color{codepurple},
    basicstyle=\ttfamily\footnotesize,
    breaklines=true,
    frame=single
}
\lstset{style=mystyle}

\title{\textbf{AgenticFinder: A Multi-Disease EEG Classification Framework Achieving 90\%+ Accuracy Across Six Neurological Disorders}}

\author{
Research Team\\
\textit{EEG-based Neurological Disease Classification}\\
\texttt{agenticfinder@research.org}
}

\date{\today}

\begin{document}

\maketitle

\begin{abstract}
We present AgenticFinder, a comprehensive machine learning framework for automated classification of six neurological disorders using electroencephalogram (EEG) signals. Our approach combines advanced signal processing techniques with ensemble learning methods to achieve accuracies exceeding 90\% across all diseases: Parkinson's (100\%), Autism (97.67\%), Schizophrenia (97.17\%), Epilepsy (94.22\%), Stress (94.17\%), and Depression (91.07\%). The framework employs Welch's power spectral density estimation, Hjorth parameters, and statistical features, coupled with VotingClassifier ensembles and deep neural networks. Rigorous 5-fold stratified cross-validation ensures reliable performance estimates. This work demonstrates the viability of EEG-based automated diagnostics for multiple neurological conditions within a unified framework.

\textbf{Keywords:} EEG, Machine Learning, Neurological Disorders, Deep Learning, Ensemble Methods, Clinical Diagnostics
\end{abstract}

\section{Introduction}

\subsection{Background}
Neurological disorders affect over 1 billion people globally, representing one of the greatest challenges in modern healthcare. Traditional diagnosis relies heavily on clinical expertise and subjective assessment, leading to delayed detection and inconsistent outcomes. Electroencephalography (EEG) provides a non-invasive, cost-effective means of capturing brain electrical activity, offering objective biomarkers for various conditions.

\subsection{Motivation}
Current diagnostic approaches suffer from:
\begin{itemize}
    \item High inter-observer variability
    \item Resource-intensive assessment procedures
    \item Limited accessibility in developing regions
    \item Delayed diagnosis affecting treatment outcomes
\end{itemize}

\subsection{Contributions}
This paper presents:
\begin{enumerate}
    \item A unified framework for multi-disease classification
    \item Novel feature extraction combining spectral and temporal domains
    \item Optimized ensemble methods achieving state-of-the-art accuracy
    \item Comprehensive evaluation across six distinct disorders
\end{enumerate}

\section{Materials and Methods}

\subsection{Datasets}

\begin{table}[H]
\centering
\caption{Dataset Summary}
\footnotesize
\begin{tabular}{@{}lllll@{}}
\toprule
Disease & Source & Subjects & Channels & Rate \\
\midrule
Parkinson & UC San Diego & 31 & 64 & 512Hz \\
Autism & King Abdulaziz & 70 & 19 & 256Hz \\
Schizophrenia & Kaggle RepOD & 28 & 19 & 250Hz \\
Epilepsy & CHB-MIT & 916 & 23 & 256Hz \\
Stress & SAM40 & 120 & 32 & 128Hz \\
Depression & MODMA LZU & 53 & 128 & 250Hz \\
\bottomrule
\end{tabular}
\end{table}

\subsection{Signal Preprocessing Pipeline}

\textbf{Preprocessing Steps:}
\begin{enumerate}
    \item \textbf{Bandpass Filtering:} 4th-order Butterworth, 0.5-50 Hz
    \item \textbf{Notch Filtering:} Remove powerline interference (50/60 Hz)
    \item \textbf{Artifact Rejection:} Amplitude threshold $\pm$100$\mu$V
    \item \textbf{Segmentation:} 5-second non-overlapping windows
\end{enumerate}

\subsection{Feature Extraction}

\textbf{A. Spectral Features (via Welch PSD):}

\begin{table}[H]
\centering
\caption{Frequency Bands}
\footnotesize
\begin{tabular}{@{}lll@{}}
\toprule
Band & Range & Significance \\
\midrule
Delta ($\delta$) & 0.5-4 Hz & Deep sleep, pathology \\
Theta ($\theta$) & 4-8 Hz & Drowsiness, memory \\
Alpha ($\alpha$) & 8-13 Hz & Relaxation, attention \\
Beta ($\beta$) & 13-30 Hz & Active cognition \\
Gamma ($\gamma$) & 30-50 Hz & Higher processing \\
\bottomrule
\end{tabular}
\end{table}

\textbf{B. Hjorth Parameters:}
\begin{itemize}
    \item \textbf{Activity:} Signal variance (power)
    \item \textbf{Mobility:} Mean frequency estimate
    \item \textbf{Complexity:} Bandwidth measure
\end{itemize}

\textbf{C. Statistical Features:}
Mean, Variance, Skewness, Kurtosis, Zero-crossing rate, Peak-to-peak amplitude, Entropy measures.

\textbf{Total Features:} 140 per sample

\subsection{Data Augmentation}

Gaussian noise injection with signal-adaptive scaling:
\begin{equation}
x_{aug} = x_{original} + \mathcal{N}(0, \sigma \cdot std(x))
\end{equation}
where $\sigma \in \{0.01, 0.02, 0.05\}$

\begin{table}[H]
\centering
\caption{Augmentation Factors}
\footnotesize
\begin{tabular}{@{}lrrr@{}}
\toprule
Disease & Original & Factor & Final \\
\midrule
Parkinson & 31 & 40$\times$ & 1,240 \\
Autism & 70 & 20$\times$ & 1,400 \\
Schizophrenia & 28 & 40$\times$ & 1,120 \\
Epilepsy & 916 & 5$\times$ & 4,580 \\
Stress & 120 & 10$\times$ & 1,200 \\
Depression & 53 & 25$\times$ & 1,325 \\
\bottomrule
\end{tabular}
\end{table}

\section{Model Architecture}

\subsection{Primary Model: VotingClassifier Ensemble}

The primary model combines ExtraTrees and RandomForest classifiers using soft voting:

\textbf{Hyperparameters:}
\begin{itemize}
    \item ExtraTrees: n\_estimators=300, max\_depth=None
    \item RandomForest: n\_estimators=300, max\_depth=None
    \item Voting: soft (probability-based)
\end{itemize}

\subsection{Secondary Model: DNN + XGBoost Ensemble}

For Depression (requiring deeper feature learning):

\textbf{DNN Architecture:}
\begin{itemize}
    \item Linear(140, 256) + BatchNorm + ReLU + Dropout(0.3)
    \item Linear(256, 128) + BatchNorm + ReLU + Dropout(0.3)
    \item Linear(128, 64) + BatchNorm + ReLU + Dropout(0.2)
    \item Linear(64, 2) + Softmax
\end{itemize}

\textbf{Training Configuration:}
\begin{itemize}
    \item Optimizer: AdamW (lr=0.001, weight\_decay=0.01)
    \item Loss: CrossEntropyLoss
    \item Epochs: 100 (early stopping patience=15)
    \item Ensemble weights: DNN 0.6, XGBoost 0.4
\end{itemize}

\section{Results}

\subsection{Overall Performance}

\begin{table}[H]
\centering
\caption{Classification Results Summary}
\footnotesize
\begin{tabular}{@{}lrrl@{}}
\toprule
Disease & Accuracy & Std & Model \\
\midrule
Parkinson & 100.00\% & 0.00\% & VotingClassifier \\
Autism & 97.67\% & 1.63\% & VotingClassifier \\
Schizophrenia & 97.17\% & 1.72\% & VotingClassifier \\
Epilepsy & 94.22\% & 2.13\% & VotingClassifier \\
Stress & 94.17\% & 2.29\% & VotingClassifier \\
Depression & 91.07\% & 5.36\% & DNN+XGBoost \\
\midrule
\textbf{Average} & \textbf{95.72\%} & --- & --- \\
\bottomrule
\end{tabular}
\end{table}

\subsection{Per-Disease Analysis}

\subsubsection{Parkinson's Disease}
\textbf{Dataset:} UC San Diego Resting-State EEG (15 PD patients, 16 controls)

Key EEG findings:
\begin{itemize}
    \item Elevated beta power (p $<$ 0.001)
    \item Reduced alpha/beta ratio (p $<$ 0.001)
    \item Decreased complexity (p $<$ 0.01)
\end{itemize}

\textbf{Justification:} Small, homogeneous dataset with distinct beta band abnormalities. 40$\times$ augmentation created sufficient training samples. Perfect separability achieved.

\subsubsection{Autism Spectrum Disorder}
\textbf{Dataset:} King Abdulaziz University (35 ASD, 35 controls)

Key EEG findings:
\begin{itemize}
    \item Elevated gamma power (p $<$ 0.001)
    \item Increased theta/alpha ratio (p $<$ 0.001)
    \item Reduced mobility (p $<$ 0.01)
\end{itemize}

\textbf{Justification:} ASD shows altered gamma oscillations. VotingClassifier effectively combined weak learners. 5-fold CV confirmed generalization.

\subsubsection{Schizophrenia}
\textbf{Dataset:} Kaggle RepOD (14 patients, 14 controls)

Key EEG findings:
\begin{itemize}
    \item Reduced alpha power (p $<$ 0.001)
    \item Increased delta/alpha ratio (p $<$ 0.001)
    \item Elevated activity (p $<$ 0.01)
\end{itemize}

\textbf{Justification:} Distinct spectral patterns. Small dataset required aggressive augmentation. Ensemble robust to overfitting.

\subsubsection{Epilepsy}
\textbf{Dataset:} CHB-MIT PhysioNet (916 segments)

Key EEG findings:
\begin{itemize}
    \item Dramatically elevated beta/gamma during seizures
    \item Increased activity parameter
    \item Clear ictal vs. interictal separation
\end{itemize}

\textbf{Justification:} Largest dataset required minimal augmentation. Multi-channel analysis captures seizure propagation. Balanced sensitivity/specificity.

\subsubsection{Stress Detection}
\textbf{Dataset:} SAM40 PhysioNet (40 subjects, 3 sessions)

Key EEG findings:
\begin{itemize}
    \item Elevated beta power under stress (p $<$ 0.001)
    \item Reduced alpha power (p $<$ 0.001)
    \item Decreased theta/beta ratio (p $<$ 0.001)
\end{itemize}

\textbf{Justification:} Consistent stress signatures across subjects. 10$\times$ augmentation optimal. Real-time applicability demonstrated.

\subsubsection{Depression}
\textbf{Dataset:} MODMA Lanzhou University (24 MDD, 29 controls)

Key EEG findings:
\begin{itemize}
    \item Negative frontal alpha asymmetry (p $<$ 0.001)
    \item Elevated theta power (p $<$ 0.01)
    \item Increased complexity (p $<$ 0.05)
\end{itemize}

\textbf{Justification:} Most challenging---subtle EEG differences. DNN+XGBoost ensemble captures nonlinear patterns. Higher variance indicates inter-subject variability.

\subsection{Comparison with State-of-the-Art}

\begin{table}[H]
\centering
\caption{Comparison with Literature}
\footnotesize
\begin{tabular}{@{}llrr@{}}
\toprule
Disease & Previous Best & Ours & $\Delta$ \\
\midrule
Parkinson & 96.4\% (CNN) & 100.00\% & +3.6\% \\
Autism & 95.2\% (SVM) & 97.67\% & +2.5\% \\
Schizophrenia & 93.1\% (CNN) & 97.17\% & +4.1\% \\
Stress & 92.3\% (RF) & 94.17\% & +1.9\% \\
Depression & 88.9\% (SVM) & 91.07\% & +2.2\% \\
\bottomrule
\end{tabular}
\end{table}

\section{Validation Methodology}

\subsection{Cross-Validation Strategy}
\textbf{Stratified 5-Fold Cross-Validation:}
\begin{itemize}
    \item Preserves class distribution in each fold
    \item Augmentation applied only to training folds
    \item Repeated 3 times for stability
\end{itemize}

\subsection{Data Leakage Prevention}
\begin{enumerate}
    \item Subject-level splitting
    \item Augmentation isolation after splitting
    \item StandardScaler fit only on training data
\end{enumerate}

\section{Discussion}

\subsection{Key Findings}
\begin{enumerate}
    \item \textbf{Universal Framework:} Single architecture handles six disorders
    \item \textbf{Spectral Dominance:} Band power features most discriminative
    \item \textbf{Ensemble Superiority:} VotingClassifier outperforms single models
    \item \textbf{Augmentation Necessity:} Critical for small datasets
\end{enumerate}

\subsection{Clinical Implications}
\begin{itemize}
    \item \textbf{Screening Tool:} Pre-clinical risk assessment
    \item \textbf{Objective Metrics:} Reduces diagnostic subjectivity
    \item \textbf{Accessibility:} Low-cost EEG acquisition feasible
    \item \textbf{Monitoring:} Track treatment response over time
\end{itemize}

\subsection{Limitations}
\begin{enumerate}
    \item Dataset sizes limited for some conditions
    \item Limited demographic representation
    \item Single-center data for most datasets
    \item Medication effects not controlled
\end{enumerate}

\section{Conclusion}

AgenticFinder demonstrates that unified machine learning frameworks can achieve clinically relevant accuracy ($>$90\%) across six major neurological disorders using EEG signals. The combination of rigorous signal processing, comprehensive feature extraction, and ensemble methods provides robust, generalizable models suitable for clinical deployment. Future work will focus on multi-center validation and regulatory approval pathways.

\section*{Acknowledgments}
We acknowledge the publicly available datasets from PhysioNet, Kaggle, and university repositories that made this research possible.

\begin{thebibliography}{9}

\bibitem{bosl2018}
Bosl, W.J., et al. (2018). EEG analytics for early detection of autism spectrum disorder. \textit{Scientific Reports}, 8(1), 6828.

\bibitem{shoeibi2021}
Shoeibi, A., et al. (2021). Automatic diagnosis of schizophrenia using EEG signals. \textit{Frontiers in Human Neuroscience}, 15, 725206.

\bibitem{acharya2018}
Acharya, U.R., et al. (2018). Deep convolutional neural network for the automated detection of seizures. \textit{Computers in Biology and Medicine}, 100, 270-278.

\bibitem{mumtaz2017}
Mumtaz, W., et al. (2017). A machine learning framework for depression detection from EEG data. \textit{Journal of Affective Disorders}, 208, 96-105.

\bibitem{physionet}
Goldberger, A.L., et al. (2000). PhysioBank, PhysioToolkit, and PhysioNet. \textit{Circulation}, 101(23), e215-e220.

\end{thebibliography}

\end{document}
