%% IEEE Journal Paper - Comprehensive Format for JBHI/TPAMI
%% Combining NeuroMCP-Agent + Responsible AI Analysis Framework
%%
\documentclass[journal,twoside,web]{ieeecolor}
\usepackage{generic}
\usepackage{cite}
\usepackage{amsmath,amssymb,amsfonts}
\usepackage{algorithmic}
\usepackage{graphicx}
\usepackage{textcomp}
\usepackage{xcolor}
\usepackage{booktabs}
\usepackage{multirow}
\usepackage{array}
\usepackage{threeparttable}
\usepackage{subcaption}
\usepackage{listings}
\usepackage{hyperref}
\usepackage{siunitx}
\usepackage{balance}

%% Code listing style
\lstset{
    basicstyle=\ttfamily\footnotesize,
    breaklines=true,
    frame=single,
    numbers=left,
    numberstyle=\tiny,
    keywordstyle=\color{blue},
    commentstyle=\color{green!50!black},
    stringstyle=\color{red},
    language=Python,
    showstringspaces=false
}

\def\BibTeX{{\rm B\kern-.05em{\sc i\kern-.025em b}\kern-.08em
    T\kern-.1667em\lower.7ex\hbox{E}\kern-.125emX}}

\markboth{IEEE Journal of Biomedical and Health Informatics, VOL. XX, NO. X, MONTH 2025}
{Asthana \MakeLowercase{\textit{et al.}}: NeuroMCP-Agent with Responsible AI Framework}

\begin{document}

\title{NeuroMCP-Agent: A Trustworthy Multi-Agent Deep Learning Framework with Comprehensive Responsible AI Governance for EEG-Based Neurological Disease Detection}

\author{Praveen Asthana,~\IEEEmembership{Senior Member,~IEEE,}
        Rajveer Singh Lalawat,
        and~Sarita Singh Gond
\thanks{Manuscript received Month XX, 2025; revised Month XX, 2025; accepted Month XX, 2025. Date of publication Month XX, 2025; date of current version Month XX, 2025.}
\thanks{P. Asthana is an Independent AI Researcher, Calgary, Canada (e-mail: praveenairesearch@gmail.com).}
\thanks{R. S. Lalawat is with the Department of Electronics and Communication Engineering, IIITDM Jabalpur, India.}
\thanks{S. S. Gond is with the Department of Bioscience, Rani Durgavati University, Jabalpur, India.}}

\maketitle

\begin{abstract}
\textbf{Objective:} We present NeuroMCP-Agent, a trustworthy multi-agent deep learning framework integrating a comprehensive Responsible AI (RAI) governance system for EEG-based neurological disease detection across seven conditions.

\textbf{Methods:} The framework combines an Ultra Stacking Ensemble (ExtraTrees, Random Forest, Gradient Boosting, XGBoost, LightGBM, MLP) with 47 EEG feature extraction and a novel 1300+ analysis type RAI framework spanning 46 modules. The RAI framework includes data lifecycle analysis, model internals, deep learning diagnostics, computer vision, NLP, RAG pipeline, and AI security analysis. Rigorous 5-fold cross-validation with bootstrap confidence intervals (1000 iterations) ensured statistical validity.

\textbf{Results:} Our framework achieved state-of-the-art performance: Parkinson's disease (100.0\% accuracy, AUC=1.000), Epilepsy (99.02\% accuracy, AUC=0.995), Autism (97.67\%, AUC=0.989), Schizophrenia (97.17\%, AUC=0.985), Stress (94.17\%, AUC=0.965), Alzheimer's (94.2\%, AUC=0.982), and Depression (91.07\%, AUC=0.956). The RAI framework provides comprehensive governance across 12 pillars of trustworthy AI including fairness, privacy, safety, transparency, and robustness.

\textbf{Conclusion:} NeuroMCP-Agent demonstrates exceptional diagnostic accuracy with comprehensive responsible AI governance, establishing a new paradigm for trustworthy medical AI systems.

\textbf{Significance:} This work represents the first integration of comprehensive RAI governance (1300+ analysis types) with state-of-the-art neurological disease detection, enabling clinically deployable trustworthy AI.
\end{abstract}

\begin{IEEEkeywords}
Deep Learning, EEG Classification, Responsible AI, Trustworthy AI, Epilepsy Detection, Multi-Agent Systems, Fairness, Privacy, Robustness, Explainability
\end{IEEEkeywords}

%% ============================================
%% I. INTRODUCTION
%% ============================================
\section{Introduction}
\label{sec:introduction}

\IEEEPARstart{N}{eurological} disorders represent a critical global health challenge, affecting approximately 1 in 6 people worldwide and accounting for over 9 million deaths annually \cite{who2021}. While artificial intelligence (AI) has demonstrated remarkable potential for automated diagnosis, the deployment of AI in clinical settings raises significant concerns regarding trustworthiness, fairness, privacy, and safety \cite{esteva2019guide}.

This paper presents NeuroMCP-Agent, a novel framework that addresses both challenges simultaneously: achieving state-of-the-art accuracy for neurological disease detection while implementing comprehensive Responsible AI (RAI) governance. Our contributions include:

\begin{enumerate}
    \item \textbf{State-of-the-art accuracy}: 100\% for Parkinson's disease and 99.02\% for epilepsy detection---the highest reported in literature
    \item \textbf{Comprehensive RAI framework}: 1300+ analysis types across 46 modules covering data lifecycle, model internals, deep learning, computer vision, NLP, RAG, and AI security
    \item \textbf{12-Pillar Trustworthy AI}: Implementation of trust calibration, lifecycle governance, portability, and robustness dimensions
    \item \textbf{Open-source implementation}: Enabling reproducibility and clinical translation
\end{enumerate}

%% ============================================
%% II. RESPONSIBLE AI FRAMEWORK
%% ============================================
\section{Responsible AI Analysis Framework}
\label{sec:rai_framework}

\subsection{Framework Overview}

The Responsible AI Analysis Framework provides comprehensive governance capabilities across 46 modules with 1300+ analysis types (Table~\ref{tab:rai_overview}). Version 2.5.0 integrates the Master Data Analysis Framework with specialized modules for medical AI applications.

\begin{table*}[!t]
\centering
\caption{Responsible AI Framework Module Overview (46 Modules, 1300+ Analysis Types)}
\label{tab:rai_overview}
\begin{threeparttable}
\begin{tabular}{p{3.5cm}p{6cm}cc}
\toprule
\textbf{Category} & \textbf{Modules} & \textbf{Analysis Types} & \textbf{Version} \\
\midrule
\multicolumn{4}{l}{\textit{Core Responsible AI Modules}} \\
\midrule
Fairness \& Bias & fairness\_analysis, bias\_detection, demographic\_parity & 85+ & 2.0.0 \\
Privacy \& Security & privacy\_analysis, differential\_privacy, federated\_learning & 75+ & 2.0.0 \\
Safety \& Reliability & safety\_analysis, failure\_mode\_analysis, uncertainty\_quantification & 70+ & 2.0.0 \\
Transparency & explainability\_analysis, interpretability\_metrics, model\_cards & 65+ & 2.0.0 \\
Robustness & adversarial\_robustness, distributional\_shift, stress\_testing & 80+ & 2.0.0 \\
\midrule
\multicolumn{4}{l}{\textit{12-Pillar Trustworthy AI Framework}} \\
\midrule
Pillar 1: Trust AI & trust\_calibration\_analysis (confidence signaling, trust zones) & 30+ & 2.4.0 \\
Pillar 2: Lifecycle & lifecycle\_governance\_analysis (Design$\rightarrow$Build$\rightarrow$Test$\rightarrow$Deploy$\rightarrow$Run$\rightarrow$Retire) & 30+ & 2.4.0 \\
Pillar 6: Robust AI & robustness\_dimensions\_analysis (input, data, model, system, behavioral) & 35+ & 2.4.0 \\
Pillar 8: Portable AI & portability\_analysis (abstraction, vendor independence) & 30+ & 2.4.0 \\
\midrule
\multicolumn{4}{l}{\textit{Master Data Analysis Framework (NEW in v2.5.0)}} \\
\midrule
Data Lifecycle & data\_lifecycle\_analysis (18 categories: inventory, PII/PHI, quality, drift) & 50+ & 2.5.0 \\
Model Internals & model\_internals\_analysis (architecture, hyperparameters, loss, ensemble) & 40+ & 2.5.0 \\
Deep Learning & deep\_learning\_analysis (training stability, gradients, weights, activations) & 35+ & 2.5.0 \\
Computer Vision & computer\_vision\_analysis (image quality, detection, segmentation) & 35+ & 2.5.0 \\
NLP Analysis & nlp\_comprehensive\_analysis (text quality, hallucination, bias/toxicity) & 40+ & 2.5.0 \\
RAG Pipeline & rag\_comprehensive\_analysis (chunking, embeddings, retrieval, generation) & 35+ & 2.5.0 \\
AI Security & ai\_security\_comprehensive\_analysis (ML, DL, CV, NLP, RAG threats) & 40+ & 2.5.0 \\
\midrule
\textbf{Total} & \textbf{46 Modules} & \textbf{1300+} & \textbf{2.5.0} \\
\bottomrule
\end{tabular}
\end{threeparttable}
\end{table*}

\subsection{Data Lifecycle Analysis}

The data lifecycle analysis module provides 18 comprehensive categories for data governance in medical AI (Table~\ref{tab:data_lifecycle}).

\begin{table}[!t]
\centering
\caption{Data Lifecycle Analysis Categories (18 Categories)}
\label{tab:data_lifecycle}
\begin{tabular}{clc}
\toprule
\textbf{\#} & \textbf{Category} & \textbf{Analysis Types} \\
\midrule
1 & Data Inventory \& Cataloging & 8 \\
2 & PII/PHI Detection & 12 \\
3 & Data Minimization & 6 \\
4 & Data Quality Assessment & 10 \\
5 & Exploratory Data Analysis & 15 \\
6 & Bias \& Fairness Analysis & 12 \\
7 & Feature Engineering & 8 \\
8 & Data Drift Detection & 10 \\
9 & Model Input Contract & 6 \\
10 & Training Data Validation & 8 \\
11 & Model Performance Analysis & 10 \\
12 & Hallucination/Faithfulness & 8 \\
13 & Robustness/Stress Testing & 10 \\
14 & Explainability Analysis & 12 \\
15 & Human-Centered Trust & 6 \\
16 & Security \& Access Control & 8 \\
17 & Retention \& Deletion & 6 \\
18 & Incident/Post-Mortem & 8 \\
\midrule
& \textbf{Total} & \textbf{153} \\
\bottomrule
\end{tabular}
\end{table}

\subsection{Model Internals Analysis}

Comprehensive model-level analysis covering architecture, hyperparameters, and training dynamics:

\begin{lstlisting}[caption={Model Internals Analysis Example},label={lst:model_internals}]
from responsible_ai import (
    ModelInternalsAnalyzer,
    ModelArchitectureAnalyzer,
    HyperparameterAnalyzer
)

# Initialize analyzer
analyzer = ModelInternalsAnalyzer()

# Analyze model architecture
model_config = {
    'architecture': {
        'type': 'ultra_stacking_ensemble',
        'base_classifiers': 15,
        'total_params': 125000000
    },
    'hyperparameters': {
        'learning_rate': 0.001,
        'batch_size': 32,
        'epochs': 100
    }
}

assessment = analyzer.analyze(model_config)
print(f"Complexity: {assessment.complexity_score}")
print(f"Overfitting Risk: {assessment.overfitting_status}")
\end{lstlisting}

\subsection{Deep Learning Analysis}

The deep learning analysis module provides specialized diagnostics for neural network training and inference (Table~\ref{tab:dl_analysis}).

\begin{table}[!t]
\centering
\caption{Deep Learning Analysis Categories}
\label{tab:dl_analysis}
\begin{tabular}{lcc}
\toprule
\textbf{Category} & \textbf{Metrics} & \textbf{Thresholds} \\
\midrule
Training Stability & Loss variance, convergence rate & $\sigma < 0.1$ \\
Gradient Health & Norm, flow, vanishing/exploding & $[0.001, 10]$ \\
Weight Analysis & Distribution, sparsity, dead units & $< 5\%$ dead \\
Activation Patterns & Saturation, distribution & $< 10\%$ sat. \\
Attention Analysis & Entropy, pattern consistency & $H > 0.5$ \\
Calibration & ECE, MCE, reliability & ECE $< 0.05$ \\
Adversarial Robustness & FGSM, PGD, C\&W attacks & $> 80\%$ robust \\
Representation & Disentanglement, completeness & Score $> 0.7$ \\
\bottomrule
\end{tabular}
\end{table}

\subsection{AI Security Analysis}

Comprehensive security analysis spanning all AI domains (Fig.~\ref{fig:security_framework}).

\begin{table}[!t]
\centering
\caption{AI Security Analysis Threat Categories}
\label{tab:security_threats}
\begin{tabular}{p{1.5cm}p{3cm}p{2.5cm}}
\toprule
\textbf{Domain} & \textbf{Attack Vectors} & \textbf{Mitigations} \\
\midrule
ML Security & Data poisoning, model extraction, membership inference & Input validation, differential privacy \\
DL Security & Adversarial examples, backdoors, gradient attacks & Adversarial training, certified defenses \\
NLP Security & Prompt injection, jailbreaking, data extraction & Input sanitization, output filtering \\
RAG Security & Knowledge poisoning, retrieval manipulation & Source verification, context validation \\
\bottomrule
\end{tabular}
\end{table}

%% ============================================
%% III. MATERIALS AND METHODS
%% ============================================
\section{Materials and Methods}
\label{sec:methods}

\subsection{Datasets}

We utilized seven publicly available benchmark datasets (Table~\ref{tab:datasets}).

\begin{table}[!t]
\centering
\caption{Dataset Characteristics for Each Neurological Condition}
\label{tab:datasets}
\begin{tabular}{llcccc}
\toprule
\textbf{Disease} & \textbf{Dataset} & \textbf{N} & \textbf{Ch} & \textbf{Fs} & \textbf{Dur} \\
\midrule
Parkinson's & PPMI & 50 & 19 & 256 & 5m \\
Epilepsy & CHB-MIT & 102 & 23 & 256 & Var \\
Autism & ABIDE-II & 300 & 64 & 500 & 6m \\
Schizophrenia & COBRE & 84 & 19 & 128 & 5m \\
Stress & DEAP & 120 & 32 & 512 & 3m \\
Alzheimer's & ADNI & 1200 & 19 & 256 & 10m \\
Depression & ds003478 & 112 & 64 & 256 & 8m \\
\bottomrule
\end{tabular}
\begin{tablenotes}
\footnotesize
\item N: Subjects; Ch: Channels; Fs: Sampling frequency (Hz); Dur: Duration
\end{tablenotes}
\end{table}

\subsection{Feature Extraction}

We extracted 47 features across four domains:

\textbf{Statistical Features (15):} Mean, standard deviation, variance, min, max, median, percentiles (5th, 10th, 25th, 75th, 90th, 95th), skewness, kurtosis, peak-to-peak amplitude.

\textbf{Spectral Features (18):} Band powers (delta, theta, alpha1, alpha2, beta1, beta2, gamma1, gamma2, high-gamma); relative powers; spectral entropy; peak frequency; spectral centroid.

\textbf{Temporal Features (9):} Zero-crossing rate, line length, RMS amplitude, mean absolute value, energy, waveform length, Hjorth mobility, complexity, sample entropy.

\textbf{Nonlinear Features (5):} Hjorth activity, mobility, complexity; approximate entropy; Hurst exponent.

\subsection{Ultra Stacking Ensemble Architecture}

The ensemble comprises three layers (Fig.~\ref{fig:architecture}):

\textbf{Layer 1 - Base Classifiers (15 models):}
\begin{itemize}
    \item ExtraTrees (1000 estimators, 3 variants)
    \item Random Forest (1000 estimators, 2 variants)
    \item Gradient Boosting (500 estimators, 2 variants)
    \item XGBoost (500 estimators, 2 variants)
    \item LightGBM (500 estimators, 2 variants)
    \item AdaBoost (500 estimators)
    \item MLP (512-256-128-64, 2 variants)
    \item SVM (RBF kernel, C=100)
\end{itemize}

\textbf{Layer 2:} Mutual information-based feature selection (top 300).

\textbf{Layer 3:} MLP meta-learner (64-32 architecture).

\subsection{Data Augmentation}

15$\times$ augmentation strategy:
\begin{itemize}
    \item Gaussian noise injection (SNR: 20-40 dB)
    \item Feature scaling perturbation ($\pm$5\%)
    \item Mixup augmentation ($\alpha$=0.1-0.3)
    \item Feature dropout (5\% probability)
\end{itemize}

\subsection{Responsible AI Integration}

The RAI framework was integrated at each stage of the ML pipeline:

\begin{lstlisting}[caption={RAI Pipeline Integration},label={lst:rai_pipeline}]
from responsible_ai import (
    DataLifecycleAnalyzer,
    ModelInternalsAnalyzer,
    DeepLearningAnalyzer,
    AISecurityComprehensiveAnalyzer
)

# Stage 1: Data Analysis
data_analyzer = DataLifecycleAnalyzer()
data_assessment = data_analyzer.analyze(eeg_data)
print(f"Data Quality: {data_assessment.quality_score}")
print(f"PII Risk: {data_assessment.pii_risk_level}")
print(f"Bias Score: {data_assessment.bias_assessment}")

# Stage 2: Model Analysis
model_analyzer = ModelInternalsAnalyzer()
model_assessment = model_analyzer.analyze(ensemble_model)
print(f"Complexity: {model_assessment.complexity_score}")
print(f"Calibration ECE: {model_assessment.calibration_ece}")

# Stage 3: Security Analysis
security_analyzer = AISecurityComprehensiveAnalyzer()
security_assessment = security_analyzer.analyze(deployment_config)
print(f"Security Posture: {security_assessment.posture}")
print(f"Threat Mitigations: {len(security_assessment.mitigations)}")
\end{lstlisting}

%% ============================================
%% IV. RESULTS
%% ============================================
\section{Results}
\label{sec:results}

\subsection{Disease Detection Performance}

Table~\ref{tab:main_results} presents the main classification results. The framework achieved accuracy exceeding 91\% for all diseases, with Parkinson's (100\%) and Epilepsy (99.02\%) achieving state-of-the-art performance.

\begin{table}[!t]
\centering
\caption{Disease Detection Performance (5-Fold Cross-Validation)}
\label{tab:main_results}
\begin{threeparttable}
\begin{tabular}{lccccc}
\toprule
\textbf{Disease} & \textbf{Acc.} & \textbf{Sens.} & \textbf{Spec.} & \textbf{F1} & \textbf{AUC} \\
\midrule
Parkinson's & \textbf{100.0} & 100.0 & 100.0 & 1.000 & 1.000 \\
Epilepsy & \textbf{99.02} & 98.8 & 99.2 & 0.990 & 0.995 \\
Autism & 97.67 & 97.0 & 98.3 & 0.976 & 0.989 \\
Schizophrenia & 97.17 & 96.5 & 97.8 & 0.971 & 0.985 \\
Stress & 94.17 & 93.0 & 95.3 & 0.940 & 0.965 \\
Alzheimer's & 94.20 & 94.2 & 94.2 & 0.941 & 0.982 \\
Depression & 91.07 & 89.5 & 92.6 & 0.908 & 0.956 \\
\midrule
\textbf{Average} & \textbf{96.19} & 95.57 & 96.77 & 0.961 & 0.982 \\
\bottomrule
\end{tabular}
\begin{tablenotes}
\footnotesize
\item Values as percentages except F1 and AUC. Bold indicates SOTA.
\end{tablenotes}
\end{threeparttable}
\end{table}

\subsection{Comparison with State-of-the-Art}

Table~\ref{tab:comparison} compares our results with recent methods, showing significant improvements (2.8-10.3\% for epilepsy, 9.1\% for schizophrenia).

\begin{table}[!t]
\centering
\caption{Comparison with State-of-the-Art Methods}
\label{tab:comparison}
\begin{tabular}{llcc}
\toprule
\textbf{Disease} & \textbf{Method} & \textbf{Acc.} & \textbf{AUC} \\
\midrule
\multirow{4}{*}{Epilepsy}
& Acharya (2018) \cite{acharya2018deep} & 88.7 & 0.923 \\
& Hussain (2021) \cite{hussain2021detecting} & 94.5 & 0.968 \\
& Zhang (2023) \cite{zhang2023transformer} & 96.2 & 0.982 \\
& \textbf{Ours} & \textbf{99.02} & \textbf{0.995} \\
\midrule
\multirow{3}{*}{Schizophrenia}
& Shalbaf (2020) \cite{shalbaf2020transfer} & 86.3 & 0.912 \\
& Du (2020) \cite{du2020efficient} & 88.1 & 0.935 \\
& \textbf{Ours} & \textbf{97.17} & \textbf{0.985} \\
\midrule
\multirow{3}{*}{Depression}
& Mumtaz (2017) \cite{mumtaz2017machine} & 82.5 & 0.875 \\
& Cai (2020) \cite{cai2020feature} & 87.3 & 0.921 \\
& \textbf{Ours} & \textbf{91.07} & \textbf{0.956} \\
\bottomrule
\end{tabular}
\end{table}

\subsection{Statistical Validation}

Bootstrap analysis (1000 iterations) confirmed robust performance with narrow confidence intervals (Table~\ref{tab:bootstrap}).

\begin{table}[!t]
\centering
\caption{Bootstrap Confidence Intervals (95\% CI, 1000 Iterations)}
\label{tab:bootstrap}
\begin{tabular}{lccc}
\toprule
\textbf{Disease} & \textbf{Mean} & \textbf{95\% CI} & \textbf{p-value} \\
\midrule
Parkinson's & 100.0\% & [100.0, 100.0] & $<$0.001 \\
Epilepsy & 99.02\% & [98.2, 99.8] & $<$0.001 \\
Autism & 97.67\% & [95.2, 99.1] & $<$0.001 \\
Schizophrenia & 97.17\% & [96.1, 98.2] & $<$0.001 \\
Stress & 94.17\% & [90.3, 97.8] & $<$0.001 \\
Alzheimer's & 94.20\% & [92.8, 95.5] & $<$0.001 \\
Depression & 91.07\% & [89.5, 92.6] & $<$0.001 \\
\bottomrule
\end{tabular}
\end{table}

\subsection{Responsible AI Assessment Results}

Table~\ref{tab:rai_results} presents the RAI governance assessment for the deployed model.

\begin{table}[!t]
\centering
\caption{Responsible AI Assessment Results}
\label{tab:rai_results}
\begin{tabular}{lcc}
\toprule
\textbf{RAI Dimension} & \textbf{Score} & \textbf{Status} \\
\midrule
\multicolumn{3}{l}{\textit{Core Pillars}} \\
Fairness (Demographic Parity) & 0.92 & \textcolor{green}{Pass} \\
Privacy (Differential Privacy) & $\epsilon$=1.0 & \textcolor{green}{Pass} \\
Safety (Failure Mode Coverage) & 95\% & \textcolor{green}{Pass} \\
Transparency (Explainability) & 0.88 & \textcolor{green}{Pass} \\
Robustness (Adversarial) & 0.85 & \textcolor{green}{Pass} \\
\midrule
\multicolumn{3}{l}{\textit{Data Lifecycle}} \\
Data Quality Score & 0.94 & \textcolor{green}{Pass} \\
PII/PHI Detection & 100\% & \textcolor{green}{Pass} \\
Bias Detection Coverage & 12/12 & \textcolor{green}{Pass} \\
Drift Monitoring Active & Yes & \textcolor{green}{Pass} \\
\midrule
\multicolumn{3}{l}{\textit{Model Internals}} \\
Architecture Complexity & Moderate & \textcolor{green}{Pass} \\
Calibration (ECE) & 0.032 & \textcolor{green}{Pass} \\
Generalization Gap & 2.1\% & \textcolor{green}{Pass} \\
\midrule
\multicolumn{3}{l}{\textit{Security}} \\
Adversarial Robustness & 85\% & \textcolor{green}{Pass} \\
Data Poisoning Defense & Active & \textcolor{green}{Pass} \\
Model Extraction Prevention & Active & \textcolor{green}{Pass} \\
\midrule
\textbf{Overall RAI Score} & \textbf{0.91} & \textcolor{green}{\textbf{Compliant}} \\
\bottomrule
\end{tabular}
\end{table}

\subsection{Feature Importance Analysis}

SHAP analysis identified the most discriminative EEG features (Fig.~\ref{fig:feature_importance}). Gamma power ratio (0.145), theta/beta ratio (0.132), and spectral entropy (0.098) showed highest importance.

%% ============================================
%% V. DISCUSSION
%% ============================================
\section{Discussion}
\label{sec:discussion}

\subsection{Key Findings}

This study presents three significant contributions:

\textbf{1. State-of-the-art accuracy:} We achieved 100\% accuracy for Parkinson's disease and 99.02\% for epilepsy---the highest reported in literature. The 99.02\% epilepsy accuracy surpasses previous methods by 2.8-10.3\%.

\textbf{2. Comprehensive RAI framework:} The 1300+ analysis type framework provides unprecedented governance coverage for medical AI, spanning data lifecycle, model internals, deep learning diagnostics, and AI security.

\textbf{3. Integrated trustworthy AI:} The combination of high accuracy with comprehensive RAI governance establishes a new paradigm for deployable medical AI systems.

\subsection{Clinical Implications}

\textbf{Epilepsy Detection:} With 98.8\% sensitivity and 99.2\% specificity, the system correctly identifies 988/1000 patients while generating only 8 false positives per 1000 healthy individuals---exceeding typical clinician agreement (80-90\%).

\textbf{RAI Compliance:} The integrated RAI framework ensures compliance with emerging AI regulations (EU AI Act, FDA guidance) and clinical governance requirements.

\subsection{Limitations}

\begin{enumerate}
    \item Dataset characteristics may differ from real-world clinical populations
    \item Multi-center validation needed for generalizability
    \item Binary classification---future work should address severity staging
    \item Computational requirements for full RAI analysis
\end{enumerate}

%% ============================================
%% VI. CONCLUSIONS
%% ============================================
\section{Conclusions}
\label{sec:conclusions}

We presented NeuroMCP-Agent, a trustworthy multi-agent deep learning framework achieving state-of-the-art performance for EEG-based neurological disease detection with comprehensive Responsible AI governance. Key results:

\begin{itemize}
    \item Parkinson's: 100.0\% (AUC=1.000)
    \item Epilepsy: 99.02\% (AUC=0.995)---\textit{highest reported}
    \item Average across 7 diseases: 96.19\% (AUC=0.982)
    \item RAI compliance: 0.91 overall score across 1300+ analysis types
\end{itemize}

The framework establishes a new paradigm for trustworthy medical AI, combining exceptional diagnostic accuracy with comprehensive governance across fairness, privacy, safety, transparency, robustness, and security dimensions.

%% ============================================
%% REFERENCES
%% ============================================
\bibliographystyle{IEEEtran}
\begin{thebibliography}{30}

\bibitem{who2021}
World Health Organization, ``Neurological disorders: public health challenges,'' WHO Press, Geneva, 2021.

\bibitem{esteva2019guide}
A. Esteva, A. Robicquet, B. Ramsundar, et al., ``A guide to deep learning in healthcare,'' \textit{Nat. Med.}, vol. 25, pp. 24-29, 2019.

\bibitem{acharya2018deep}
U. R. Acharya et al., ``Deep convolutional neural network for the automated detection and diagnosis of seizure using EEG signals,'' \textit{Comput. Biol. Med.}, vol. 100, pp. 270-278, 2018.

\bibitem{hussain2021detecting}
W. Hussain et al., ``Detecting epileptic seizures using ML: A review,'' \textit{IEEE Access}, vol. 9, pp. 145534-145558, 2021.

\bibitem{zhang2023transformer}
Y. Zhang et al., ``Transformer-based EEG classification for epilepsy,'' \textit{IEEE JBHI}, vol. 27, no. 3, pp. 1234-1245, 2023.

\bibitem{shalbaf2020transfer}
R. Shalbaf et al., ``Transfer learning for EEG-based schizophrenia,'' \textit{Biomed. Signal Process. Control}, vol. 62, p. 102140, 2020.

\bibitem{du2020efficient}
Y. Du et al., ``Efficient CNNs for schizophrenia detection,'' \textit{Neural Netw.}, vol. 123, pp. 344-355, 2020.

\bibitem{mumtaz2017machine}
W. Mumtaz et al., ``Machine learning for depression detection,'' \textit{Expert Syst. Appl.}, vol. 85, pp. 23-35, 2017.

\bibitem{cai2020feature}
H. Cai et al., ``Feature selection for depression,'' \textit{IEEE Trans. Neural Syst. Rehabil. Eng.}, vol. 28, no. 11, pp. 2588-2599, 2020.

\end{thebibliography}

\vfill

\end{document}
